%! TEX program = xelatex
\documentclass{highlights}

\usepackage{lipsum}

\title{Highlights of \textit{Fundamentals of Microelectronics}}
\author{Jaeho Lee}
\date{\today}

\begin{document}
\maketitle

\section{Introduction to Microelectronics}
\begin{concept}{Voltage Gain}
  \emph{Voltage gain $A_v$} in a voltage amplifier:
  \begin{equation}
    A_v = \frac{v_\mathrm{out}}{v_\mathrm{in}}.
  \end{equation}
  Expressed in decibels (dB):
  \begin{equation}
    A_v|_{\mathrm{dB}} = 20 \log \frac{v_\mathrm{out}}{v_\mathrm{in}}.
  \end{equation}
\end{concept}

\begin{concept}{Kirchoff's Laws}
  \emph{The Kirchoff Current Law (KCL).} The sum of all currents flowing \emph{into} a node is zero:
  \begin{equation}
    \sum_j I_j = 0.
  \end{equation}
  \tcbline
  \emph{The Kirchoff Voltage Law (KVL).} The sum of voltage drops around any closed loop in a circuit is zero:
  \begin{equation}
    \sum_j V_j = 0.
  \end{equation}
\end{concept}

\begin{concept}{Thevenin and Norton Equivalents}
  \emph{Thevenin's theorem.} A linear one-port network can be replaced with a voltage source in series with an impedance.
  The \emph{equivalent voltage $v_\mathrm{Thev}$} can be calculated by leaving the port open;
  The \emph{equivalent impedance $Z_\mathrm{Thev}$} can be determined by setting all independent voltage and current sources to zero.
  \tcbline
  \emph{Norton's theorem.} A linear one-port network can be replaced with a current source in parallel with an impedance.
  The \emph{equivalent current $i_\mathrm{Nor}$} can be obtained by shorting the port;
  The \emph{equivalent impedance $Z_\mathrm{Nor}$} can be determined by setting all independent voltage and current sources to zero.
  \tcbline
  Note that $Z_\mathrm{Thev} = Z_\mathrm{Nor}$.
\end{concept}

\section{Basic Physics of Semiconductors}
\begin{concept}{Lorem Ipsum}
  \lipsum[1]
\end{concept}
\end{document}
